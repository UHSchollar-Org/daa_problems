\documentclass{article}
\usepackage{amsfonts}
\usepackage{amsthm}
\usepackage{amsmath}
\usepackage{graphicx}
\usepackage{listings}
\setlength{\parindent}{0cm}
\lstset{
    numbers=left,
    numbersep=15pt,
    numberfirstline = false,
    breaklines=true,
}
\begin{document}

\title{Tercer Problema DAA}
\author{Karlos Alejandro Alfonso Rodríguez \\ Karel Camilo Manresa Leon}
\maketitle
\newpage

\section*{Problema}
    
    La Pregunta
    Estaba Karlos pasando (tal vez perdiendo) el tiempo cuando viene Karel y le
    hace una pregunta. Karlos quería responder que "no" a la pregunta, pero
    Karel le dijo que no era tan fácil, que la respuesta a esa pregunta iba a
    depender de un conjunto de pequeñas preguntas de "si" o "no" que este
    tenía. Luego, con las respuestas a esas preguntas de "si" o "no" armó una
    expresión booleana conformada por negaciones, expresiones "and",
    expresiones "or" e implicaciones de las pequeñas preguntas. La respuesta
    de la pregunta grande dependería de la expresión booleana que a la vez
    dependía de las pequeñas preguntas. Ayude a Karlos a encontrar si existe
    una distribución de respuestas a las pequeñas preguntas que le permitan
    responder que "no" a la pregunta grande.

\section*{Problema en términos matemáticos}

    Dada una expresión booleana, se quiere saber para que asignación de las variables
    la expresión evalúa $False$. 

\section*{Análisis del problema}

    Sea $P^*$ el problema \textbf{La Pregunta}, dicho problema es de optimización, ya que pide buscar 
    una distribución de valores para las variables de la expresión booleana. Sea $P$ la versión de decisión
    de $P^*$, es decir, se quiere saber si existe una alguna distribución de valores para las variables de la expresión
    tal que dicha expresión evalúe $False$. Demostremos que $P$ es $NP-completo$, luego $P^*$ será $NP-Hard$.

    \subsection*{NP-completitud de P}
        \subsubsection*{NP}

            Sea $A$ un algoritmo que puede verificar $P$. Una de las entradas de $A$ es una 
            codificación estándar de una fórmula lógica $F$ (literales, negaciones, expresiones and, 
            expresiones or, implicaciones). La otra entrada es un certificado correspondiente 
            a una asignación de un valor booleano a cada una de las variables de la fórmula $F$. 

            El algoritmo $A$ se construye de la siguiente manera. Se genera una $FNC$ de $\urcorner F$, luego por 
            cada cláusula de $FNC$ se verifica el valor de sus literales en la asignación. 
            Si una cláusula $C$ contiene al menos un literal que 
            su asignación sea positiva, entonces dicha cláusula será positiva, en caso de que todos los literales
            de $C$ tengan asignaciones negativos, entonces $C$ será negativa. Si todas las cláusulas de $FNC$ 
            son positivas, entonces $FNC$ será positiva; en caso contrario $FNC$ será negativa. La respuesta de $A$
            será la satisfacibilidad de $FNC$. 

            Generar la $FNC$ de $\urcorner F$ se realiza en tiempo polinomial utilizando el algoritmo de $Tseitin$. Luego 
            el recorrido por cada cláusula y cada literal se realiza en $O(cant\_clausulas)$, por lo tanto $A$ es polinomial. 
            Luego el algoritmo $A$ tiene tiempo polinomial. Luego $P$ es $NP$.

        \subsubsection*{Reducción}

            Se seleccionó el problema $NP-completo$ SAT (Boolean satisfiability problem) como problema conocido para 
            aplicarle una reducción polinómica al problema $P$.
            
            El problema SAT, tiene como entrada una expresión lógica en forma normal conjuntiva ($FNC$). El problema $P$, 
            tiene como entrada una expresión lógica cualquiera $F$, podemos transformar $F$ en una entrada de SAT 
            (en tiempo polinomial) aplicando el algoritmo de $Tseitin$ (explicado a continuación) sobre $\urcorner F$. 
            Entonces $FNC$ será la entrada de SAT.
            
            Sea $S$ una salida del problema SAT, $S$ será la salida correcta de $P$, ya que como $FNC \cong \urcorner F$, entonces 
            $FNC = True \Rightarrow  F = False$ y $FNC = False \Rightarrow F = True$. Luego $S$ soluciona el problema $P$.

        \subsubsection*{Algoritmo de Tseitin}
    
\section*{Soluciones para el problema}

    Podemos clasificar el problema en dos casos según el número 
    de literales por cláusula. Si cada cláusula tiene a lo sumo dos literales, es una variante de 2-SAT. 
    En este caso, el problema puede resolverse eficientemente en tiempo polinomial.

    Por otro lado, si cada cláusula contiene tres o más literales, la complejidad del problema 
    aumenta significativamente y solo se podrá resolver en tiempo exponencial.

    \subsection*{Fuerza bruta}

        La solución más sencilla para el problema es la fuerza bruta. En este caso, se evalúa 
        cada posible asignación de valores a las variables de la expresión y se verifica si la 
        expresión es verdadera o falsa. Si la expresión es verdadera, se devuelve la asignación 
        de valores que la hace verdadera. En caso contrario, se devuelve que la expresión no es 
        satisfacible.

        \subsubsection*{Pseudocódigo}

\begin{lstlisting}[language=Python]
def exhaustive_enumeration(expresion):
    cnf = to_CNF(expresion)
    n = len(expresion.vars)
    vars = [0] * n
    pos = 0
    result = generate_assignments(vars, pos, n, len(cnf), cnf)
    return result

def generate_assignments(vars, pos, n, m, cnf):
    if pos == n:
        if evaluate(vars, cnf):
            return vars
        else:
            return None
    else:
        vars[pos] = 0
        if generate_assignments(vars, pos + 1, n, m, cnf):
            return vars
        
        vars[pos] = 1
        if generate_assignments(vars, pos + 1, n, m, cnf):
            return vars
        
        return None
\end{lstlisting}

        El algoritmo de fuerza bruta tiene una complejidad exponencial en el peor caso. En 
        particular, si la fórmula tiene $n$ variables y $m$ cláusulas, entonces el número
        de posibles asignaciones de valores de verdad es $2^n$, lo que significa que se necesitan 
        $2^n$ iteraciones para evaluar todas las posibles asignaciones. En cada iteración, el 
        algoritmo debe evaluar la fórmula lógica para determinar si la asignación actual satisface la 
        fórmula, esto implica recorrer todos los literales de cada cláusula, para 
        verificar si dicha asignación satisface la fórmula. Luego la complejidad temporal 
        sería $O(2^n * |F|)$.
    
    \subsection*{Algoritmo DPLL}
        El algoritmo DPLL (Davis-Putnam-Logemann-Loveland) es un algoritmo de resolución de satisfacibilidad booleana (SAT) que se utiliza 
        para determinar si una fórmula lógica proposicional es satisfacible.Este algoritmo trabaja de manera recursiva, y se basa en 
        la siguiente idea: si una fórmula lógica es satisfacible, 
        entonces debe ser posible encontrar una asignación de valores de verdad a las variables proposicionales que satisfaga la fórmula. 
        Por lo tanto, el algoritmo DPLL busca una asignación de valores de verdad que satisfaga la fórmula, y utiliza técnicas de poda para 
        reducir el espacio de búsqueda.
            
        El algoritmo DPLL comienza con una fórmula lógica en forma CNF, y utiliza dos tipos de reglas para simplificar la fórmula:
            
        \begin{itemize}
            \item Regla de unidad: Si una cláusula contiene solo un literal, entonces ese literal debe ser verdadero para 
            satisfacer la cláusula. Por lo tanto, se puede asignar un valor de verdad a esa variable proposicional y eliminar 
            todas las cláusulas que contienen ese literal.

            \item Regla de purga: Si una variable proposicional aparece solo en una polaridad en todas las cláusulas restantes, 
            entonces se puede asignar un valor de verdad a esa variable de manera que satisfaga todas las cláusulas que contienen 
            esa polaridad. Luego, se pueden eliminar todas las cláusulas que contienen esa variable.
        \end{itemize}

        Después de aplicar estas reglas, el algoritmo DPLL verifica si la fórmula se ha simplificado lo suficiente como para que sea 
        trivialmente satisfacible o insatisfacible.  

        Si ninguna de las reglas anteriores se puede aplicar, entonces el algoritmo DPLL elige una variable proposicional 
        arbitraria y prueba dos casos diferentes: asignarle un valor de verdad verdadero y un valor de verdad falso.
        Luego, el algoritmo DPLL aplica recursivamente las reglas de simplificación a la fórmula resultante en cada caso. 
        Si alguna de las ramas recursivas devuelve una solución satisfactoria, entonces el algoritmo DPLL devuelve esa solución.
        Si ambas devuelven contradicciones, entonces es insatisfacible.

            \subsubsection*{Pseudocódigo}
\begin{lstlisting}[language=Python]
def dpll(expresion):
    cnf = to_CNF(expresion)
    n = len(expresion.vars)
    vars = [0] * n
    assigns = [False] * n
    return dpll_rec(vars, assigns, n, len(cnf), cnf)

def dpll_rec(vars, assigns, n, len(cnf), cnf):
    if evaluate(vars, cnf):
        return vars
    
    literal = select_next_literal(vars, assigns)
    assigns[literal-1] = True
    values = select_literal_values(random)

    for value in values:
        vars[literal-1] = value
        new_formula, valid_solution = unit_propagation(literal, value, vars, assigns, cnf)
    
        if valid_solution:
            result, valid_solution = dpll_rec(vars, assigns, n, len(cnf), new_formula)
            
            if valid_solution:
                return result
    
    return None
\end{lstlisting}

                Métodos utilizados:
                \begin{itemize}
                    \item \textbf{select\_next\_literal:} Selecciona el proximo literal al que se le va a asignar un valor. En este caso se selecciono
                    dicho literal de manera aleatoria. No obstante existen heurísticas más complejas para esta selección que aprovechan la cantidad de 
                    variables y particularidades de la fórmula, obteniendo resultados con mayor velocidad.
                    \item \textbf{select\_literal\_values:} Selecciona el valor que tomará el literal, al igual que el método
                     \textbf{select\_next\_literal} lo hace de manera aleatoria.
                    \item \textbf{unit\_propagation:} Dado el valor y la varible seleccionados, evalúa dicha asignación y la propaga por la fórmula, 
                    verificando asignaciones triviales.
                \end{itemize}

                La complejidad temporal del algoritmo DPLL depende del tamaño de la fórmula booleana, es decir, del 
                número de variables y de cláusulas en la fórmula. En el peor caso, el número total de posibles asignaciones de 
                valores de verdad a las variables es $2^n$ ($n=cantidad\_variables$). Además las operaciones de simplificación
                (reglas de purga y unidad) se aplican por cada cláusula por tanto esto influye también en la complejidad temporal
                del algoritmo, siendo esta $O(2^n*m)$.

                Obsérvese que DPLL tiene la misma complejidad temporal que el algoritmo por fuerza bruta; no obstante, en la práctica 
                DPLL es significativamente más eficiente que la fuerza bruta debido a las técnicas de poda que utiliza (reglas de purga y unidad).
    
    \subsection*{Algoritmo genético}
        
        En el contexto del problema SAT, los algoritmos genéticos pueden ser útiles para encontrar
        una solución satisfactoria al problema. A diferencia de los algoritmos de backtracking y los algoritmos
        basados en DPLL, los algoritmos genéticos no garantizan encontrar una solución óptima 
        al problema SAT. Sin embargo, pueden ser útiles en casos donde la fórmula FNC es muy grande o compleja
        y se requiere de una solución satisfactoria en un tiempo razonable.

        El algoritmo genético implementado recibe una fórmula en FNC como entrada y sigue los siguientes pasos:

        \begin{enumerate}
            \item Representación de cromosomas:  Cada posible asignación de valores de verdad a las 
            variables proposicionales en la fórmula CNF se representa como un cromosoma. 
            En una representación binaria, cada cromosoma puede ser una cadena de bits, donde cada bit 
            representa el valor de verdad de una variable proposicional.
            \item Evaluación de la aptitud: La calidad de cada cromosoma se evalúa utilizando una 
            función de fitness que mide la satisfacibilidad de la fórmula CNF para esa 
            asignación de valores de verdad. Esta función devuelve un valor de aptitud que 
            indica el número de cláusulas que se satisfacen para esa asignación de valores de verdad.
            \item Creación de la población inicial: Se crea una población inicial de cromosomas aleatorios, 
            donde cada cromosoma representa una posible asignación de valores de verdad a 
            las variables proposicionales.
            \item Selección: Se seleccionan los cromosomas más aptos (mayor fitness) de la población actual para la 
            reproducción. Se seleccionan los $\frac{n}{2}$ mejores cromosomas como parte de la nueva población y 
            para generar $\frac{n}{2}$ nuevos cromosomas. 
            \item Operador de cruce: Se combinan dos cromosomas para producir 
            un nuevo cromosoma que comparta características de ambos progenitores.
            \item Operador de mutación: Se aplica el operador de mutación (con una probabilidad definida al inicio del algoritmo) 
            a los nuevos cromosomas 
            generados para introducir variabilidad en la población. El operador de mutación cambia 
            aleatoriamente uno o más bits en el cromosoma.
            \item Evaluación de la aptitud: Se evalúa la aptitud de los nuevos cromosomas utilizando 
            la función de fitness.
            \item Sustitución: Se sustituyen los cromosomas menos aptos de la población actual con 
            los nuevos cromosomas generados.
            \item Comprobación de la condición de terminación: Se comprueba si se ha encontrado una 
            solución satisfactoria o si se ha alcanzado un número de generaciones. 
            Si se ha encontrado una solución satisfactoria, se devuelve el mejor cromosoma como solución.
            \item Repetición del proceso: Si no se ha encontrado una solución satisfactoria, 
            se repiten los pasos 4 a 9 hasta que se cumpla la condición de terminación.
        \end{enumerate}



\end{document}