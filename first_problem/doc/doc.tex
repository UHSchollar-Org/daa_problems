\documentclass{article}
\usepackage{amsfonts}
\usepackage{amsthm}
\usepackage{amsmath}
\usepackage{graphicx}
\usepackage{listings}
\setlength{\parindent}{0cm}
\lstset{
    numbers=left,
    numbersep=15pt,
    numberfirstline = false,
    breaklines=true,
}

\begin{document}

\title{Primer Problema DAA}
\author{Karlos Alejandro Alfonso Rodríguez \\ Karel Camilo Manresa Leon}
\maketitle

\newpage

\section*{Problema}

Leandro es profesor de programación. En sus ratos libres, le gusta divertirse con las 
estadísticas de sus pobres estudiantes reprobados. Los estudiantes están separados en n grupos. 
Casualmente este año, todos los estudiantes reporbaron alguno (uno y solo uno) de los dos exámenes 
finales $P$ (POO) y $R$ (Recursividad).

Esta tarde, Leandro decide entretenerse separando a los estudiantes
suspensos en conjuntos de tamaño k que cumplan lo siguiente: En un mismo conjunto, todos los estudiantes 
son del mismo grupo $i$ $(1 \leq i \leq n)$ o suspendieron por el mismo exámen $P$ o $R$.

Conociendo el grupo y prueba suspensa de cada estudiante y el tamaño de los conjuntos, 
ayude a Leandro a saber cuántos conjuntos de estudiantes suspensos puede formar.

\section*{Problema en términos matemático}

Se tiene una lista $G$, donde $G[i]$ indica el conjunto al que pertenece el entero $i$.
Además se tiene otra lista binaria $F$, donde $F[i]$ indica la presencia o no de una característica en 
el entero $i$. De modo tal que para un entero $i$ se tiene $G[i]$ y $F[i]$, que 
indican: conjunto al que pertenece $i$ y presencia o no de una característica para $i$, respectivamente.

Dado un entero $k \geq 1$ y las listas $G$ y $F$, se desea conocer cuantos conjuntos de tamaño $k$ se pueden formar 
de modo que $\forall i,j \in k_t$ se cumpla $G[i] = G[j] \vee F[i] = F[j]$.

Sea:
\begin{itemize}
    \item $|G|$ = n.
    \item $P$ conjunto de todos los elementos que cumplen $F[i] = 1$.
    \item $R$ conjunto de todos los elementos que cumplen $F[i] = 0$.
\end{itemize}

\section*{Primer enfoque}

El primer enfoque pensado para la solución del problema es hacer un $Backtrack$, que
compruebe todas las posibles formas de construir subconjuntos disjuntos
válidos ($G[i] = G[j] \vee F[i] = F[j]$) de tamaño máximo $k$. Luego se selecciona como 
respuesta la forma que contenga la mayor cantidad de subconjuntos de tamaño $k$.

La cantidad máxima de subconjuntos de tamaño $k$ que se pueden realizar con $n$ elementos
es $\lfloor \frac{n}{k} \rfloor$. En este caso debemos añadir dos conjuntos más ya que 
el resto resultante de hacer parte entera por debajo, en el peor caso tiene elementos de 
ambas característica, y deben agruparse en conjuntos diferentes. 

\subsection*{Correctitud}

Esta solución es correcta ya que comprueba todas las formas posibles de construir los subconjuntos
y se queda con la mejor. 

\subsection*{Pseudocódigo}

\begin{lstlisting}[language = Python]
def backtrack(element, G, F, k, subsets):
    if element == |G|:
        return KSizeSets(subsets, k)
    
    sol = 0
    for i from 0 to |subsets|:
        if |subsets[i]| == k:
            continue
        
        if Insert(subsets[i], element):
            sol = max(sol, backtrack(element+1, G, F, k, subsets))
            subsets[i].Remove(element)
            if sol == |G|//k:
                break
    
    return sol

\end{lstlisting}

\subsection*{Complejidad temporal}

% El costo temporal de ejecutar el caso de parada (2-3) es:
% \begin{eqnarray*}
%     &=& O(|subsets|)\\
%     &=& O(\lfloor \frac{n}{k} \rfloor + 2)\\
%     &=& O(\frac{n}{k})
% \end{eqnarray*}

%Analicemos el costo del ciclo (6-14). Dicho ciclo se ejecuta $|subsets|$ veces,
%lo cual es $O(\frac{n}{k})$, luego el llamado a $Insert$ tiene un costo de $O(k)$,
%ya que $k$ es el tamaño máximo que puede alcanzar cada conjunto. El costo de 
%eliminar el elemento del subconjunto es $O(k)$. Por tanto el ciclo tiene un costo
%de $O(\frac{n}{k}k) = O(n)$.



%Por cada elemento se realiza un llamado recursivo de la función inicial, entonces 
%tendríamos $n$ llamados con costo $O(n)$ cada uno, lo que equivale $O(\prod_{i=1}^{n} n) = O(n^n)$.

En el algoritmo recursivo anterior tenemos que $f(n) = 2k$ ya que $Insert$ y $Remove$
recorren $k$ elementos cada uno. Luego sea $m = |G|$, tenemos que se realizan $\lfloor \frac{m}{k}\rfloor$
llamados a la recursividad. Cada llamado se realiza con un elemento menos, por lo que 
$T(n) = \lfloor \frac{m}{k} \rfloor T(n-1) + 2k$. 

Entonces: 
\begin{eqnarray*}
    T(n) &=& 2k + \lfloor \frac{m}{k} \rfloor(2k + \lfloor \frac{m}{k}\rfloor(2k + \lfloor \frac{m}{k}\rfloor(\dots)))\\
    T(n) &=& (\frac{m}{k})^m + 2(\frac{m}{k})^{m-1} + \dots + 2k \\
    T(n) &\leq& (\frac{m}{k})^m + 2(\frac{m}{k})^m + \dots + 2k^m = O(\frac{m^m}{k^m})\\
\end{eqnarray*}


\section*{Segundo enfoque}

El segundo enfoque pensado para la solución del problema es hacer un algoritmo greedy.
Primeramente se intenta crear todos los subconjuntos posibles priorizando la existencia o no
de la característica; o sea, la solución será la cantidad de subconjuntos de tamño $k$ que 
se pueden formar con elementos que cumplan la característica más la cantidad de subconjuntos 
que se pueden formar con elementos que no la cumplan. Luego se comprueba si con los elementos 
restantes se puede formar otro subconjunto basado en el conjunto inicial al que pertenecen dichos
elementos.

\subsection*{Correctitud}

Para demostrar la correctitud de esta solución, se debe demostrar que siempre 
que se creen los subconjuntos basados en las características, se obtendrá un total de 
$\lfloor \frac{n}{k} \rfloor - 1$ o $\lfloor \frac{n}{k} \rfloor$ subconjuntos de tamaño 
$k$. En caso de obtenerse $\lfloor \frac{n}{k} \rfloor - 1$ subconjuntos y la respuesta 
correcta ser $\lfloor \frac{n}{k} \rfloor$, se debe probar que siempre hay una manera de 
armar un subconjunto basado en el conjunto inicial de los elementos.

Sea $p = |P|$ y $r = |R|$, además $p'$ y $r'$ son la cantidad de conjuntos de tamaño $k$
que se pueden formar con los elementos de $P$ y $R$ respectivamente.
Demostremos que $\lfloor \frac{n}{k} \rfloor - 1 \leq p' + r' \leq \lfloor \frac{n}{k} \rfloor$
\begin{proof}
    
    Se sabe que
    \begin{equation*}
        p + r = n
    \end{equation*}
    luego si aplicamos el algoritmo de división sobre $p$ y $r$ con $k$ de divisor se obtiene
    \begin{equation*}
        kp' + p_r + kr' + r_r = n
    \end{equation*}
    donde $p'$ y $r'$ son precisamente la cantidad de subconjuntos de tamaño $k$ que se pueden
    formar con los conjuntos $P$ y $R$ respectivamente. Si multiplicamos la ecuación anterior 
    por $k$ obtenemos
    \begin{equation*}
        p' + r' + \frac{p_r + r_r}{k} = \frac{n}{k}
    \end{equation*}
    si aplicamos la función parte entera por debajo a la ecuación
    \begin{equation*}
        \lfloor p' + r' + \frac{p_r + r_r}{k} \rfloor = \lfloor \frac{n}{k}\rfloor
    \end{equation*}
    y como $p'$ y $r'$ son enteros se cumple que $\lfloor p' \rfloor = p'$ y 
    $\lfloor r' \rfloor = r'$, por tanto 
    \begin{equation*}
        p' + r' + \lfloor \frac{p_r + r_r}{k} \rfloor = \lfloor \frac{n}{k} \rfloor
    \end{equation*}

    Si $0 \leq \lfloor \frac{p_r + r_r}{k} \rfloor \leq 1$ entonces 
    $\lfloor \frac{n}{k} \rfloor - 1 \leq p' + r' \leq \lfloor \frac{n}{k} \rfloor$. Demostremos 
    entonces que $0 \leq \lfloor \frac{p_r + r_r}{k} \rfloor \leq 1$.

    \begin{proof}
        Como $p_r$ y $r_r$ son enteros positivos, su suma será un entero positivo, por tanto al hacer
        $\frac{p_r + r_r}{k}$ se obtiene un número positivo ya que $k$ es un entero positivo también. 
        Luego al aplicarle parte entera por debajo se obtiene un entero positivo, luego 
        $0 \leq \lfloor\frac{p_r+r_r}{k}\rfloor$.

        Se tiene además que 
        \begin{eqnarray*}
            p_r &<& k\\
            r_r &<& k\\
            p_r + r_r &<& 2k\\
            \frac{p_r+r_r}{k} &<& 2\\
            \lfloor \frac{p_r+r_r}{k} \rfloor &\leq& 1
        \end{eqnarray*}
        Luego $0 \leq \lfloor \frac{p_r + r_r}{k} \rfloor \leq 1$.
    \end{proof}
    Entonces se cumple que $\lfloor \frac{n}{k} \rfloor - 1 \leq p' + r' \leq \lfloor \frac{n}{k} \rfloor$.
\end{proof}

Ahora se debe demostrar que siempre que la respuesta arrojada por el algoritmo anterior
sea $\lfloor \frac{n}{k} \rfloor - 1$ y la correcta sea $\lfloor \frac{n}{k} \rfloor$ se puede 
formar un nuevo subconjunto con el resto de los conjuntos $P$ y $R$.

La forma de hacer esto pensada fue: por cada conjunto inicial $g$ si se cumple que 
$min(g[0], r_r) + min(g[1], p_r) \geq k$ entonces se puede formar otro subconjunto válido de tamaño
$k$ en ese grupo, y será el último (en caso contrario se puedieran formar al menos dos nuevos subconjuntos, 
lo cual es una contradicción ya que la cantidad máxima posible es $\lfloor \frac{n}{k} \rfloor$), 
donde $g[0]$ es la cantidad de elementos del grupo $g$ que no cumplen la característica 
y $g[1]$ los de $g$ que la cumplen. A pesar de que esta forma de armar un nuevo subconjunto es válida, no es 
factible aplicarla luego de haber agrupado por característica, ya que no necesariamente con los elementos 
restantes se podrá lograr. A continuación un caso que prueba esto:

Sea $G = [1,1,2,2,3,3]$, $F = [0,1,0,1,0,1]$ y $k = 2$. Para este caso es fácil comprobar que la respuesta 
es $sol = 3$. Si seguimos el algoritmo planteado, primeramente agrupamos por característica, obteniendo 2 
subconjuntos de tamaño 2. Luego si se intenta formar un nuevo subconjunto basado en los grupos iniciales no 
es posible ya que cada resto pertenece a un grupo distinto. Por tanto este enfoque no es válido. 
\newpage
\subsection*{Pseudocódigo}

\begin{lstlisting}[language = Python]
def solve(G, F, k):
    sol = 0

    p,r = Count 1s,0s in F
    
    sol += p//k
    sol += r//k

    p_r = p%k
    r_r = r%k

    if p_r + r_r < k:
        return sol
    
    fbs = FeaturesBySets(G, F)

    for g in fbs:
        if min(g[0], p_r) + min(g[1], r_r) >= k:
            sol+=1
    
            return sol

\end{lstlisting}

La función $FeaturesBySets(G, F)$ crea un diccionario donde dado un grupo inicial
devuelve la cantidad de elementos que tiene que cumplen la característica y 
los que no la cumplen.

\subsection*{Complejidad temporal}

La complejidad temporal de ejecutar la línea \textbf{4} es $O(|F|) = O(|G|) = O(n)$.
Luego ejecutar las líneas \textbf{2, 6-13} tiene costo $O(1)$. El diccionario $fbs$ se 
puede crear en tiempo $O(|G|) = O(n)$; y hacer el ciclo de las líneas \textbf{17-19} 
tiene costo $O(cantGrupos)$ y como $cantGrupos \leq n$, entonces tiene costo $O(n)$.
Finalmente $T(n) = 3O(n) = O(n)$.

\section*{Tercer enfoque}



\subsection*{Correctitud}

\subsection*{Pseudocódigo}

\subsection*{Complejidad temporal}

\section*{Tester}

\subsection*{Generador de casos de pueba}

\subsubsection*{Pseudocódigo}

\end{document}

