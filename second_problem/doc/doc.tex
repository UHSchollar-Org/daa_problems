\documentclass{article}
\usepackage{amsfonts}
\usepackage{amsthm}
\usepackage{amsmath}
\usepackage{graphicx}
\usepackage{listings}
\setlength{\parindent}{0cm}
\lstset{
    numbers=left,
    numbersep=15pt,
    numberfirstline = false,
    breaklines=true,
}
\begin{document}

\title{Segundo Problema DAA}
\author{Karlos Alejandro Alfonso Rodríguez \\ Karel Camilo Manresa Leon}
\maketitle
\newpage
\section*{Problema}
Lázaro presidente del PCC.

Han pasado 20 años desde que Lázaro se graduó de Ciencias de la Computación 
(haciendo una muy buena tesis) y las vueltas de la vida lo llevaron a convertirse en 
el presidente del Partido Comunista de Cuba. Una de sus muchas responsabilidades 
consiste en visitar zonas remotas. En esta ocasión debe visitar una ciudad campestre 
de Pinar del Río.

También han pasado 20 años desde que Marié consiguió su título en MATCOM. 
Tras años de viaje por las grandes metrópolis del mundo, en algún punto decidió 
que prefería vivir una vida tranquila, aislada de la urbanización, en una tranquila 
ciudad de Pinar del Río. Las vueltas de la vida quisieron que precisamente Marié fuera 
la única universitaria habitando la ciudad que Lázaro se dispone a visitar.

Los habitantes de la zona entraron en pánico ante la visita de una figura tan importante 
y decidieron reparar las calles de la ciudad por las que transitaría Lázaro. 
El problema está en que nadie sabía qué ruta tomaría el presidente y decidieron 
pedirle ayuda a Marié.

La ciudad tiene $n$ puntos importantes, unidos entre sí por calles cuyos tamaños se conocen. 
Se sabe que Lázaro comenzará en alguno de esos puntos $s$ y terminará el viaje en otro $t$. 
Los ciudadanos quieren saber, para cada par $s$, $t$, cuántas calles participan en algún 
camino de distancia mínima entre $s$ y $t$.

\section*{Problema en términos matemáticos}
Dado un grafo no dirigido y ponderado, donde no hay aristas de costo negativo. Se desea saber 
para cada par de vértices $s$ y $t$, cuantas aristas participan en algún camino de costo mínimo
entre $s$ y $t$.

\section*{Modelación}
Supongamos que para cada par de vétices $s$ y $d$ tenemos el conjunto $U_{s,d}$, que contiene
los vértices que participan en algún camino de costo mínimo entre $s$ y $d$. Entonces, si para cada
vértice $v \in U_{s,d}$, tenemos la cantidad de aristas que llegan en un camino de 
costo mínimo entre $s$ y $v$ (partiendo de $s$), tendríamos la cantidad de aristas que llegan a
cada vértice de un camino de costo mínimo entre $s$ y $d$, llamémosle $W_{s,v}$ a este conjunto. 
Por tanto, para $s$ y $d$, la cantidad de aristas que participan en algún camino de 
costo mínimo es $\sum_{v \in U_{s,d}}W_{s,v}$.

Luego para resolver el problema basta con calcular los conjuntos $U_{s,d}$ y $W_{s,v}$ para cada
par de vértices $s$ y $d$, y efectuar $\sum_{v \in U_{s,d}}W_{s,v}$.

\subsection*{Conjunto $W_{s,v}$}
Sea $F$ la matriz de distancias obtenida luego de aplicar el algoritmo de Floyd-Warshall. 


\section*{Solución}




\end{document}