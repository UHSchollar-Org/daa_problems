\documentclass{article}
\usepackage{amsfonts}
\usepackage{amsthm}
\usepackage{amsmath}
\usepackage{graphicx}
\usepackage{listings}
\setlength{\parindent}{0cm}
\lstset{
    numbers=left,
    numbersep=15pt,
    numberfirstline = false,
    breaklines=true,
}
\begin{document}

\title{Segundo Problema DAA}
\author{Karlos Alejandro Alfonso Rodríguez \\ Karel Camilo Manresa Leon}
\maketitle
\newpage
\section*{Problema}
Lázaro presidente del PCC.

Han pasado 20 años desde que Lázaro se graduó de Ciencias de la Computación 
(haciendo una muy buena tesis) y las vueltas de la vida lo llevaron a convertirse en 
el presidente del Partido Comunista de Cuba. Una de sus muchas responsabilidades 
consiste en visitar zonas remotas. En esta ocasión debe visitar una ciudad campestre 
de Pinar del Río.

También han pasado 20 años desde que Marié consiguió su título en MATCOM. 
Tras años de viaje por las grandes metrópolis del mundo, en algún punto decidió 
que prefería vivir una vida tranquila, aislada de la urbanización, en una tranquila 
ciudad de Pinar del Río. Las vueltas de la vida quisieron que precisamente Marié fuera 
la única universitaria habitando la ciudad que Lázaro se dispone a visitar.

Los habitantes de la zona entraron en pánico ante la visita de una figura tan importante 
y decidieron reparar las calles de la ciudad por las que transitaría Lázaro. 
El problema está en que nadie sabía qué ruta tomaría el presidente y decidieron 
pedirle ayuda a Marié.

La ciudad tiene $n$ puntos importantes, unidos entre sí por calles cuyos tamaños se conocen. 
Se sabe que Lázaro comenzará en alguno de esos puntos $s$ y terminará el viaje en otro $t$. 
Los ciudadanos quieren saber, para cada par $s$, $t$, cuántas calles participan en algún 
camino de distancia mínima entre $s$ y $t$.

\section*{Problema en términos matemáticos}
Dado un grafo no dirigido y ponderado, donde no hay aristas de costo negativo. Se desea saber 
para cada par de vértices $s$ y $t$, cuantas aristas participan en algún camino de costo mínimo
entre $s$ y $t$.

\section*{Modelación}
Supongamos que para cada par de vétices $s$ y $d$ tenemos el conjunto $U_{s,d}$, que contiene
los vértices que participan en algún camino de costo mínimo entre $s$ y $d$. Entonces, si para cada
vértice $v \in U_{s,d}$, tenemos la cantidad de aristas que llegan en un camino de 
costo mínimo entre $s$ y $v$ (partiendo de $s$), tendríamos la cantidad de aristas que llegan a
cada vértice de un camino de costo mínimo entre $s$ y $d$, llamémosle $W_{s,v}$ a este conjunto. 
Por tanto, para $s$ y $d$, la cantidad de aristas que participan en algún camino de 
costo mínimo es $\sum_{v \in U_{s,d}}W_{s,v}$.

Luego para resolver el problema basta con calcular los conjuntos $U_{s,d}$ y $W_{s,v}$ para cada
par de vértices $s$ y $d$, y efectuar $\sum_{v \in U_{s,d}}W_{s,v}$.

\subsection*{Como obtener $W_{s,v}$}
Sea $F$ la matriz de distancias obtenida luego de aplicar el algoritmo de Floyd-Warshall. 
Sea $e$ una arista del grafo. %que participa en algún camino de costo mínimo. 
Sea $u$ un vértice 
del grafo, $v$ y $w$ los vértices que forman la arista $e$. Podemos saber si $e$ participa en 
un camino de costo mínimo de $u$ a $v$ si $F[u,w] + peso(e) = F[u,v]$ (para el caso de $w$ 
si $F[u,v] + peso(e) = F[u,w]$). Demostremos esto.

\subsubsection*{Lema}
Sea $F$ la matriz resultante de aplicar Floyd-Warshall en un grafo no dirigido y ponderado.
Sea $u$, $v$ y $w$ vértices del grafo, y $e$ una arista que une a $v$ y $w$.
Si $F[u,v] + peso(e) = F[u,w]$ entonces $e$ participa en algún camino de costo mínimo entre
$u$ y $w$.

\begin{proof}
    Si $F[u,v] + peso(e) = F[u,w]$ entonces el algoritmo de Floyd-Warshall encontró un camino 
    de costo mínimo entre $u$ y $v$ con costo $F[u,v]$ y un camino de costo mínimo entre 
    $u$ y $w$ con costo $F[u,w]$. Por tanto, como $e = <v,w>$, si $e$ no participara en algún camino de costo
    mínimo entre $u$ y $w$, entonces el algoritmo de Floyd-Warshall habría encontrado un camino con costo 
    menor que $F[u,w]$, lo cual es una contradicción. Luego $e$ participa en algún camino de costo mínimo entre
    $u$ y $w$.
\end{proof}

Entonces para obtener $W_{s,v}$ para cada $s$ y $v$ basta con iterar por las aristas y vértices del grafo 
comprobando lo anterior.

\subsection*{Como obtener $U_{s,d}$}
Sea $F$ la matriz de distancias obtenida luego de aplicar el algoritmo de Floyd-Warshall.
Sea $u$, $v$ y $w$ vértices del grafo. Si $F[u,w] + F[w,v] = F[u,v]$ entonces $w$ participa en algún camino
de costo mínimo entre $u$ y $v$. Demostremos esto.

\subsubsection*{Lema}
Sea $F$ la matriz resultante de aplicar Floyd-Warshall en un grafo no dirigido y ponderado.
Sea $u$, $v$ y $w$ vértices del grafo. Si $F[u,w] + F[w,v] = F[u,v]$ entonces $w$ participa en algún camino
de costo mínimo entre $u$ y $v$.

\begin{proof}
    $F[u,w]$ es el costo de un camino de costo mínimo de $u$ a $w$, 
    $F[w,v]$ es el costo de un camino de costo mínimo de $w$ a $v$, y si la suma del costo de ambos caminos es 
    igual al costo de un camino de costo mínimo entre $u$ y $v$, entonces existe un camino $u...w...v$ que tiene costo
    mínimo, por tanto $w$ participa en algún camino de costo mínimo de $u$ a $v$. 
\end{proof}

\section*{Solución}
Teniendo en cuenta lo anterior, para cada par de vértices $s$ y $d$, se puede obtener $U_{s,d}$ y $W_{s,v}$ ($v \in V$), y
como se planteaba al inicio de la modelación del problema, calcular la solución para $[s,d]$ sería efectuar
$\sum_{v \in U_{s,d}}W_{s,v}$.

\subsection*{Pseudocódigo}
\begin{lstlisting}[language = Python]
    def solve(graph):
        F = Floyd-Warshall(graph)
        W = {u -> {v -> int}}
        solution = {u -> {v -> int}}
        
        for e in edges:
            for u in nodes:
                if F[u,e.v1] + e.weight == F[u,e.v2]:
                    W[u,e.v2] += 1
                if F[u,e.v2] + e.weight == F[u,e.v1]:
                    W[u,e.v1] += 1
        
        for u in nodes:
            for v in nodes:
                for w in nodes:
                    if F[u,w] + F[w,v] == F[u,v]:
                        solution[u,v] += W[u,w]
        
        return solution
\end{lstlisting}

\section*{Complejidad temporal}
Para calcular la complejidad temporal del algoritmo se analizarán las complejidades 
de los bloques de código \textbf{\textit{2-4}}, \textbf{\textit{6-11}} y \textbf{\textit{13-17}}.

El bloque de código \textbf{\textit{2-4}} tiene complejidad $O(|V|^3)$ ya que el costo de ejecutar 
Floyd-Warshall es $O(|V|^3)$ y el costo de inicializar $W$ y $solution$ es 
$O(|V|^2)$ y $O(|V|^3) + O(|V|^2) = O(|V|^3)$. 

En el bloque de código \textbf{\textit{6-11}} se itera por todas las aristas y vértices del grafo, 
por tanto tiene complejidad $O(EV)$ (el interior del ciclo es $O(1)$). En caso de ser denso el grafo, 
se cumple que $|E| \approx |V|^2$ por lo que el costo sería $O(|V|^3)$.

Luego en el bloque \textbf{\textit{13-17}} se realiza un triple $for$ por los vértices del grafo, 
por tanto tiene costo $O(|V|^3)$(el interior del ciclo es $O(1)$).

Por tanto la complejidad temporal del algoritmo es:
\begin{align}
    O(|V|^3) + O(|V|^2) + O(|V|^3) = O(|V|^3).
\end{align}

\end{document}